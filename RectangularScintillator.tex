\documentclass[a4paper]{article}
\usepackage [english]{babel}
\usepackage{hyperref}
%
\topmargin -60pt \oddsidemargin -0.4mm \evensidemargin -5.4mm
\setlength{\textheight}{257mm} \setlength{\textwidth}{165mm}
\setlength{\marginparsep}{8mm} \setlength{\marginparwidth}{18mm}
%
\begin{document}
\section{General information}
This library implements model of rectangular shaped scintillator for Monte Carlo simulation.
This model considers the scintillator to have constant refraction parameter.
The distribution of photon emission time and emitted photons wavelengths are given.
The absorption coefficient is defined as a function on photon wavelength.
\\
Photosensitive surfaces can be optically glued to different surfaces of the scintillator. 
The parameter of glue efficiency that means decreasing of photon's reflecting-back probability in the place of contact can be changed.
\\
Photosensitive surfaces may have registration probability depending on photon wavelength.
Photons that are registered can be processed by signal former algorithms that can be connected to this photosensor.
The implemented algorithms can process signal time (depending on the time of photons registration) and amplitude (depending on photons count).
\\
The signals can be further processed with combinations of algorithms implemented in this library or by user.
\section{Scintillator}
Class \textbf{RectangularScintillator} defined in $rectscin.h$ implements generating emitted photons and tracing them inside the scintillator.
For creating the instance of this class one should give the geometry of the scintillator as\\
$std::vector<std::pair<double,double>>$\\
where each pair corresponds to a different dimension and the elements in the pair correspond to minimum and maximum coordinates occupied by the scintillator.
The distribution of photon's emission time and wavelength are given as $RandomValueGenerator<double>$ that is declared in $math\_h/randomfunc.h$ (in submodule).
Refraction parameter is given as $double$. 
Absorption coefficient is given as $std::function<double(double)>$ because it depends on photon wavelength.\\
Method $Surface(dimension,side)$ returns the object that represents corresponding rectangular surface and allows to connect scintillator with photosensor or anything else that is designed for this.
The $dimension$ parameters means that the surface is defined by fixing corresponding coordinate to it's minimum value for left side and maximum value for right side.
surface geometry is given the same way but excluding the dimension that has fixed coordinate.
\section{Photosensor}
Class $PhotoSensitiveSurface$ defined in $sensitive.h$ contains implementation of photosensor.
It requires surface geometry (see above how it's defined), glue efficiency parameter given as $double$ between 0 and 1 and quantum efficiency defined as $std::function<double(double)>$ that depends on photon wavelength and returns value between 0 and 1.
It's recommended to use $Photosensor$ function for creating the instances of this class.
Here's the example of using photosensors:\\
$RectangularScintillator~scintillator(\{...geometry...\},...other parameters...);$\\
$auto~sensor=Photosensor(\{...surface~geometry...\},...other parameters...);$\\
$scintillator.Surface(dimension,RectDimensions::Left)>>sensor;$
\section{Signal formers}
All classes declared in $photon2signal.h$ inheriting $PhotonTimeAcceptor$ and $SignalProducent$ are used for modeling signals created by photosensors.
$PhotonTimeAcceptor$ is an interface for accepting registered photons and $SignalProducent$ implements transfering signal (defined as $double$ value) for further analysis.
After creating and configuring such object can be connected to $PhotoSensitiveSurface$ instance:\\
$auto~timesignal=TimeSignal(\{make\_pair(0,0.5),make\_pair(1,0.5)\});//average~of~1st~and~2nd~photon$\\
$scintillator.Surface(...)>>(sensor>>timesignal);$
\section{Signal analyse}
File $signal\_statistics.h$ contains classes inheriting $SignalAcceptor$ and using for analysis of simulated events.
Example:\\
$auto~statistics=make\_shared<SignalStatictics>();$\\
$scintillator.Surface(...)>>(sensor>>(timesignal>>statistics));$\\
or such way:\\
$scintillator.Surface(...)$\\
$>>(sensor1>>(make\_shared<SignalStatictics>()>>statistics1))$\\
$>>(sensor2>>(make\_shared<SignalStatictics>()>>statistics2));$\\
or another way:\\
$scintillator.Surface(...one...)>>(sensor1>>(make\_shared<SignalStatictics>()>>statistics1));$\\
$scintillator.Surface(...another...)>>(sensor2>>(make\_shared<SignalStatictics>()>>statistics2));$
\section{More complicated systems}
File $signal\_processing.h$ contains declarations of classes used for more complicated signals processing.
Classes for processing one signal inherit $SignalAcceptor$ and $SignalProducent$ at the same time.
So they can be connected to another $SignalProducent$ and another $SignalAcceptor$ instances can be connected to this classes.\\
There are also interfaces $AbstractMultiInput$ and $AbstractMultiOutput$ that allow to create classes that allow to process several signals at the same time. You can see sources of example applications to see how they can be used.
\end{document}